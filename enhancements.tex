\documentclass[11pt]{article}
\usepackage{amsmath}
\usepackage{amsfonts}
\usepackage{amsthm}
\usepackage[utf8]{inputenc}
\usepackage[margin=0.75in]{geometry}

\title{CSC111 Winter 2024 Project 1}
\author{Dimitrios Gkiokmema, Maya Edri}
\date{\today}

\begin{document}
\maketitle

\section*{Enhancements}


\begin{enumerate}

\item Describe your enhancement \#1 here
	\begin{itemize}
	\item Brief description of what the enhancement is (if it's a puzzle, also describe what steps the player must take to solve it): 
 
 We enhanced our text adventure game by introducing a new puzzle that requires the player to decipher a code to access a specific location. The player needs to find clues in different locations, gather information, and use it to decode the passcode for a locked area.  Success grants access to the last item in the game.
 
\textbf{Steps to Solve:}
        \begin{enumerate}
            \item The player reaches a location with a Puzzle, represented by the PuzzleLocation class.
            \item The player is prompted to enter a passcode to solve the puzzle.
            \item The player has a limited number of attempts to enter the correct passcode.
            \item If the correct passcode is entered, the player successfully solves the puzzle and gains access to the last item in the game.
        \end{enumerate}

	\item Complexity level (choose from low/medium/high): Medium
 
	\item Reasons you believe this is the complexity level (e.g. mention implementation details, how much code did you have to add/change from the baseline, what challenges did you face, etc.)
 Implementing this enhancement involved moderate complexity. We had to create a new class, \texttt{PuzzleLocation}, that inherits from the \texttt{Location} class but includes an additional attribute \texttt{puzzle\_code}. We also added a new method, \texttt{solve\_puzzle}, to the \texttt{Player} class, allowing the player to attempt to decipher the passcode within a limited number of tries. The complexity increased due to the need for coordinating interactions between the player, and combining pieces of information from different locations.

	% Feel free to add more subheadings if you need
	\end{itemize}

% Uncomment below section if you have more enhancements; copy-paste as many times as needed
\item Describe your enhancement here \#2
	\begin{itemize}
	\item Basic description of what the enhancement is:
	Extra action: back
	This allows the player to undo the last move recorded.
	Any variables or locations that were changed in the last turn are returned to what they were before the last turn
	\item Complexity level (low/medium/high): low
	\item Reasons you believe this is the complexity level (e.g. mention implementation details)
	At first we thought that this would be a difficult thing to implement, but was way easier in the end.
	All we did was assign a str to a variable that describes the actions taken in a turn.
	If the player types back, that variable is interpreted and the opposite happens
	For example, if you picked up an item and then typed back, the item is dropped
	% Feel free to add more subheadings if you feel the need
	\end{itemize}

\item Describe your enhancement here \#3
	\begin{itemize}
	\item Basic description of what the enhancement is:
	A new item and location descriptions (or levels) were added
	\item Complexity level (low/medium/high): low
	\item Reasons you believe this is the complexity level (e.g. mention implementation details)
	All we had to implement these was add stuff to the map, locations, and items files
	Our code in adventure and game_data took care of the rest
	Thus, this had a low complexity level because all we had to do was write more in the files
	% Feel free to add more subheadings if you feel the need
	\end{itemize}

\item Describe your enhancement here \#4
	\begin{itemize}
	\item Basic description of what the enhancement is:
	A new action to view tasks like where to deposit the items
	Typing objectives prints where every item has to be deposited and reminds you that the safe needs to be opened
	To win the game, every single one of these conditions must be met
	\item Complexity level (low/medium/high): low
	\item Reasons you believe this is the complexity level (e.g. mention implementation details)
    When adventure.py is ran, it creates a dict that displays the deposit locations of every item
    that is not already deposited in its correct location.
    This was not to difficult to code, all we had to do was loop through a list that contained all items in the game,
    check if they are at the correct location, and if not print where they need to go
	% Feel free to add more subheadings if you feel the need
	\end{itemize}
 
%\item Describe your enhancement here
%	\begin{itemize}
%	\item Basic description of what the enhancement is:
%	\item Complexity level (low/medium/high):
%	\item Reasons you believe this is the complexity level (e.g. mention implementation details 
%	% Feel free to add more subheadings if you feel the need
%	\end{itemize}

\end{enumerate}


\section*{Extra Gameplay Files}

If you have any extra \texttt{gameplay\#.txt} files, describe them below.

\end{document}
